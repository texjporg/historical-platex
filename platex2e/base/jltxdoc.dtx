% \iffalse meta-comment
% $Id: jltxdoc.dtx,v 1.2 1995/04/21 08:46:05 ken-na Exp ken-na $
%
% Copyright 1993-1995 the LaTeX3 project and any individual authors
% listed elsewhere in this file.  All rights reserved.
% 
% For further copyright information see the file legal.txt, and any
% other copyright notices in this file.
% 
% This file is part of the LaTeX2e system.
% ----------------------------------------
%   This system is distributed in the hope that it will be useful,
%   but WITHOUT ANY WARRANTY; without even the implied warranty of
%   MERCHANTABILITY or FITNESS FOR A PARTICULAR PURPOSE.
% 
%   For error reports concerning UNCHANGED versions of this file no more
%   than one year old, see bugs.txt.
% 
%   Please do not request updates from us directly.  Primary
%   distribution is through the CTAN archives.
% 
% 
% IMPORTANT COPYRIGHT NOTICE:
% 
% You are NOT ALLOWED to distribute this file alone.
% 
% You are allowed to distribute this file under the condition that it is
% distributed together with all the files listed in manifest.txt.
% 
% If you receive only some of these files from someone, complain!
% 
% Permission is granted to copy this file to another file with a clearly
% different name and to customize the declarations in that copy to serve
% the needs of your installation, provided that you comply with
% the conditions in the file legal.txt.
% 
% However, NO PERMISSION is granted to produce or to distribute a
% modified version of this file under its original name.
%  
% You are NOT ALLOWED to change this file.
% 
% 
% \fi
%
% \iffalse
%
%<*!platex>
%<class>\NeedsTeXFormat{LaTeX2e}
%<class>\ProvidesClass{ltxdoc}
%<class>         [1994/05/27 v2.0n Standard LaTeX documentation class]
%</!platex>
%<*platex>
%<class>\NeedsTeXFormat{pLaTeX2e}
%<class>\ProvidesClass{jltxdoc}
%<class>         [1996/03/01 v2.0n-p Standard pLaTeX documentation class]
%</platex>
%
%<*driver>
\documentclass{jltxdoc} % This file use `jltxdoc.cls' : platex
\GetFileInfo{jltxdoc.cls} % This file use `jltxdoc.cls' : platex
\providecommand\dst{\expandafter{\normalfont\scshape docstrip}}
\title{The file \texttt{ltxdoc.dtx} for use with 
      \LaTeXe.\thanks{This file has version
           number \fileversion, dated \filedate.}\\[2pt]
      It contains the code for \texttt{ltxdoc.cls}}
\date{\filedate}
\author{David Carlisle}
\begin{document}
\maketitle
 \DocInput{jltxdoc.dtx} % This file is not `ltxdoc.dtx' : platex
\end{document}
%</driver>
%
% \fi
%
% \CheckSum{442} % Original file's check sum is `376'
%
% \changes{v2.0i}{1994/04/29}{Update the documentation.}
%
% \section{Documentation of the \LaTeX\ sources}
%
% This class file is designed for documenting the \LaTeX\ source files.
% You may however find it generally useful as a class for typesetting
% the documentation of files  produced in `doc' format.
%
% Each documented file in the standard distribution comes with extension
% |dtx|. The appropriate class package or initex file will be extracted
% from the source by the docstrip system. Each |dtx| file may be
% directly processed with \LaTeXe, for example
%\begin{verbatim}
% latex2e docclass.dtx
%\end{verbatim}
% would produce the documentation of the Class and package interface.
%
% Each file that is used in producing the \LaTeXe\ format (ie not
% including the standard class and packages) will be printed together in
% one document if you \LaTeX\ the file |sources2e.tex|. This has the
% advantage that one can produce a full index of macro usage across all
% the source files.
%
% If you need to customise the typesetting of any of these files, there
% are two options:
% \begin{itemize}
% \item You can use \dst\ with the module `driver' to extract a small
% \LaTeX\ file that you may edit to use whatever class or package
% options you require, before inputting the source file.
% \item You can create a file |ltxdoc.cfg|. This configuration file will
% be read whenever the |ltxdoc| class is used, and so can be used to
% customise the typesetting of all the source files, without having to
% edit lots of small driver files.
%\end{itemize}
%
% The second option is usually more convenient. various possibilities
% are discussed in the next section.
%
% \section{Customisation}
%
% The simplest form of customisation is to pass more options to the
% |article| class which is loaded by |ltxdoc|. For instance if you wish
% all the documentation to be formated for A4 paper, add the following
% line to |ltxdoc.cfg|:
%\begin{verbatim}
%<*!platex>
% \PassOptionsToClass{a4paper}{article}
%</!platex>
%<*platex>
% \PassOptionsToClass{a4paper}{jarticle}
%</platex>
%\end{verbatim}
%
% All the source files are in two parts, separated by |\StopEventually|.
% The first part (should) contain `user' documentation. The second part
% is a full documented listing of the source code. The |doc| package
% provides the command |\OnlyDescription| which suppresses the code
% listings. This may also be used in the configuration file, but as the
% |doc| package is read later, you must delay the execution of
% |\OnlyDescription| until after the |doc| package has been read. The
% simplest way is to use |\AtBeginDocument|. Thus you could put the
% following in your |ltxdoc.cfg|.
%\begin{verbatim}
% \AtBeginDocument{\OnlyDescription}
%\end{verbatim}
%
%
% If the full source listing |sources2e.tex| is processed, then an index
% and change history are produced by default, however indices are not
% normally produced for individual files.
%
% As an example, consider |ltclass.dtx|, which contains the sources for
% the new class and package interface commands.  With no |cfg|
% file, a 19~page document is produced. With the above configuration
% a slightly more readable document (4~pages) is produced.
%
% Conversely, If you really want to read the source listings in detail,
% You will want to have an index. Again the index commands provided by
% the doc package may be used, but their execution must be delayed.
%\begin{verbatim}
% \AtBeginDocument{\CodelineIndex\EnableCrossrefs}
% \AtEndDocument{\PrintIndex}
%\end{verbatim}
%
% The |doc| package writes index files to be sorted using MakeIndex with
% the |gind| style, so one would then use a command such as
%\begin{verbatim}
% makeindex -s gind.ist ltclass.idx
%\end{verbatim}
% and re-run \LaTeX.
%
% Similarly to print a Change history, you would add
%\begin{verbatim}
% \AtBeginDocument{\RecordChanges}
% \AtEndDocument{\PrintChanges}
%\end{verbatim}
% to |ltxdoc.cfg|, and use  MakeIndex with a comand such as
%\begin{verbatim}
% makeindex -s gglo.ist -o ltclass.gls ltclass.glo
%\end{verbatim}
%
% Finally if you do not want to list all the sections of |source2e.tex|,
% you can use  |\includeonly| in the |cfg| file:
%\begin{verbatim}
% \includeonly{ltvers,ltboxes}
%\end{verbatim}
%
% \StopEventually
%
%
% \section{Options}                       
%
%    \begin{macrocode}
%<*class>
\DeclareOption{a5paper}{\@latexerr{Option not supported}%
   {}}
%    \end{macrocode}
%
%    \begin{macrocode}
\DeclareOption*{%
%<!platex>    \PassOptionsToClass  {\CurrentOption}{article}}
%<platex>    \PassOptionsToClass  {\CurrentOption}{jarticle}}
%    \end{macrocode}
%
% \section{Configuration}
% Input a local configuration file, if it exists.
%    \begin{macrocode}
\InputIfFileExists{ltxdoc.cfg}
           {\typeout{*************************************^^J%
                     * Local config file ltxdoc.cfg used^^J%
                     *************************************}}
           {}
%    \end{macrocode}
%    
%
% \section{Option Processing}
%
%    \begin{macrocode}
\ProcessOptions
%    \end{macrocode}
%
% \section{Loading article and doc}
%
%    \begin{macrocode}
%<!platex>\LoadClass{article}
%<platex>\LoadClass{jarticle}
%    \end{macrocode}
%
%    \begin{macrocode}
\RequirePackage{doc}
%    \end{macrocode}
%
% Make \verb+|+ be a `short verb' character, but not in the document
% preamble, where an active character may interfere with packages that
% are loaded.
%    \begin{macrocode}
\AtBeginDocument{\MakeShortVerb{\|}}
%    \end{macrocode}
%
%    \begin{macrocode}
\CodelineNumbered
\DisableCrossrefs
%    \end{macrocode}
%
% Increase the text width slightly so that width the standard fonts
% 72 columns of code may appear in a |macrocode| environment.
% \changes{v2.0c}{1994/03/15}{Set \cmd{\textwidth}}
%    \begin{macrocode}
\setlength{\textwidth}{355pt}
%    \end{macrocode}
%
% Increase the marginpar width slightly, for long command names.
% \changes{v2.0l}{1994/05/25}{Increase \cmd{\marginparwidth}}
%    \begin{macrocode}
\addtolength\marginparwidth{5pt}
%    \end{macrocode}
%
% Silently substitute normal tt for bold. 
% \changes{v2.0m}{1994/05/26}{Add font substition}
%    \begin{macrocode}
\DeclareFontShape{OT1}{cmtt}{bx}{n}
  {<->ssub * cmtt/m/n}{}
%    \end{macrocode}
%
%    \begin{macrocode}
\setcounter{StandardModuleDepth}{1}
%    \end{macrocode}
%
% \section{Useful abbreviations}
%
% |\cmd{\foo}| Prints |\foo| verbatim. It may be used inside moving
% arguments. |\cs{foo}| also prints |\foo|, for those who prefer that
% syntax. (This second form may even be used when |\foo| is |\outer|).
% \begin{macro}{\cmd}
% \changes{v2.0k}{1994/05/21}{New definition, so \cmd\{ works.}
% \begin{macro}{\cs}
% \changes{v2.0d}{1994/03/17}{Add \cs{cs}}
%    \begin{macrocode}
\def\cmd#1{\cs{\expandafter\cmd@to@cs\string#1}}
\def\cmd@to@cs#1#2{\char\number`#2\relax}
\DeclareRobustCommand\cs[1]{\texttt{\char`\\#1}}
%    \end{macrocode}
% \end{macro}
% \end{macro}
%
% |\star| prints |*| for use in documenting star-forms of commands.
% |\marg{text}| prints \marg{text}, `mandatory argument'.
% |\oarg{text}| prints \oarg{text}, `optional argument'.
% |\parg{te,xt}| prints \parg{te,xt}, `picture mode argument'.
% \changes{v2.0d}{1994/03/17}{Add \cs{marg}}
% \changes{v2.0h}{1994/04/28}{Add \cs{parg}}
%    \begin{macrocode}
\providecommand\star{%
  \ttfamily*}
\providecommand\marg[1]{%
  {\ttfamily\char`\{}{\em#1\/}{\ttfamily\char`\}}}
\providecommand\oarg[1]{%
  {\ttfamily[}{\em#1\/}{\ttfamily]}}
\providecommand\parg[1]{%
  {\ttfamily(}{\em#1\/}{\ttfamily)}}
%    \end{macrocode}
%
% \section{Old Comments}
%
% The \LaTeXe\ sources contain a lot of code inherited from
% \LaTeX2.09. The comments in this code were not designed to be
% typeset, and do not contain the necessary \LaTeX\ markup. The
% \texttt{oldcomments} environment typesets these comments,
% automatically sensing when any control sequence appears, and
% implicitly adding the |\verb|. This procedure does not produce
% particularly beautiful pages, but it allows us to fully document new
% sections, and have some form of typeset comments on all the old
% code. 
% \changes{v2.0e}{1994/03/18}{Use a fixed font.}
%
% Scan control names and put them in tt.
% will actually (incorrectly) scan past |\\| but this does not matter as
% this is almost never followed by a letter in practice.  
% (ie |\\foo|) would put |foo| in |\ttfamily|.
%    \begin{macrocode}
\def\oc@scan#1{%
  \ifx\oc@bslash#1%
                      \egroup\let\next\oc@bslash\else
  \ifcat a\noexpand#1%
                      #1\let\next\oc@scan\else
  \ifx\oc@percent#1%
                      \def\next{\char`\%\egroup}%
  \else
                      #1\let\next\egroup
  \fi\fi\fi\next}
%    \end{macrocode}
%
%    \begin{macrocode}
\def\oc@bslash{\bgroup\oc@ttf\char`\\\oc@scan}%
%    \end{macrocode}
%
%    \begin{macrocode}
\def\oc@verb#1{%
  \catcode`#1\active
  \uccode`\~`#1%
  \uppercase{\def~{{\oc@ttf\char`#1}}}}
%    \end{macrocode}
%
%    \begin{macrocode}
\begingroup
  \obeyspaces%
  \catcode`\/=\catcode`\\
  /catcode`/\/active
  /catcode`<=/catcode`{%
  /catcode`>=/catcode`}%
  /catcode`/{/active%
  /catcode`/}/active%
  /gdef/oldc< \end{oldcomments}>%
  /gdef/begmac<    \begin{macrocode}>%
  /gdef/obs</def <</oc@ttf/ >>>%
/endgroup%
%    \end{macrocode}
%
%    \begin{macrocode}
\begingroup
  \catcode`\/=\catcode`\\
  \catcode`\\=13
  /catcode`/|=/catcode`/%
  /catcode`/%=13
  /gdef/oldcomments{|
    /makeatletter
    /let/do/oc@verb/dospecials
    /frenchspacing/@vobeyspaces/obs
    /raggedright
    /oc@verb/>|
    /oc@verb/<|
    /let\/oc@bslash
    /let%/oc@percent
    /obeylines
    /parindent/z@
    /ttfamily/expandafter/let/expandafter/oc@ttf/the/font
    /rmfamily
    /hfuzz/maxdimen
    }
/endgroup
%    \end{macrocode}
%
%    \begin{macrocode}
\begingroup
  \sloppy%
  \obeylines%
  \gdef\oc@percent#1^^M{%
    \ifvmode%
    \def\commentline{#1}%
    \ifx\commentline\oldc%
    \end{oldcomments}%
    \else%
    \ifx\commentline\begmac%
    \begin{macrocode}%
    \else%
    \leavevmode%
    #1^^M%
    \fi\fi%
    \else%
    {\oc@ttf\char`\%}#1^^M%
    \fi}%
\endgroup%
%    \end{macrocode}
%
%
% \section{DocInclude}
%
%    \begin{macrocode}
\@addtoreset{CodelineNo}{part}
%    \end{macrocode}
%
% \begin{macro}{\DocInclude}
% More or less exactly the same as |\include|, but uses |\DocInput|
% on a |dtx| file, not |\input| on a |tex| file.
% \changes{v2.0b}{1994/03/14}{Rename from \cmd{\docinclude}}
% \changes{v2.0f}{1994/03/25}{Use \cmd{\part}}
%    \begin{macrocode}
\def\partname{File}
%    \end{macrocode}
%
%    \begin{macrocode}
\def\DocInclude#1{%
  \relax
  \clearpage
  \docincludeaux
  \def\currentfile{#1.dtx}%
  \ifnum\@auxout=\@partaux
    \@latexerr{\string\include\space cannot be nested}\@eha
  \else \@docinclude#1 \fi}
\def\@docinclude#1 {\clearpage
\if@filesw \immediate\write\@mainaux{\string\@input{#1.aux}}\fi
\@tempswatrue\if@partsw \@tempswafalse\edef\@tempb{#1}\@for
\@tempa:=\@partlist\do{\ifx\@tempa\@tempb\@tempswatrue\fi}\fi
\if@tempswa \let\@auxout\@partaux \if@filesw
\immediate\openout\@partaux #1.aux
\immediate\write\@partaux{\relax}\fi
\part{#1.dtx}%
  {\let\ttfamily\relax
  \xdef\filekey{\filekey, \thepart={\ttfamily\currentfile}}}%
\DocInput{#1.dtx}%
\clearpage
\@writeckpt{#1}\if@filesw \immediate\closeout\@partaux \fi
\else\@nameuse{cp@#1}\fi\let\@auxout\@mainaux}
%    \end{macrocode}
%
%    \begin{macrocode}
\gdef\codeline@wrindex#1{\if@filesw
        \immediate\write\@indexfile
            {\string\indexentry{#1}%
            {\filesep\number\c@CodelineNo}}\fi}%
%    \end{macrocode}
% \end{macro}
%
%    \begin{macrocode}
\let\filesep\@empty
%    \end{macrocode}
%
%
%  \begin{macro}{\aalph}
% Special  form of |\alph| as currently |source2e.tex|
% includes more than 26 files
% \changes{v2.0n}{1994/05/27}{Use uppercase letters, for makeindex}.
%    \begin{macrocode}
\def\aalph#1{\@aalph{\csname c@#1\endcsname}}
\def\@aalph#1{%
  \ifcase#1\or a\or b\or c\or d\or e\or f\or g\or h\or i\or
         j\or k\or l\or m\or n\or o\or p\or q\or r\or s\or
         t\or u\or v\or w\or x\or y\or z\or A\or B\or C\or
         D\or E\or F\or G\or H\or I\or J\or K\or L\or M\or
         N\or O\or P\or Q\or R\or S\or T\or U\or V\or W\or
         X\or Y\or Z\else\@ctrerr\fi}
%    \end{macrocode}
%  \end{macro}
%
% \begin{macro}{\docincludeaux}
% \changes{v2.06}{1994/03/31}{Use \cmd{\footnotesize} in file key.}
% \changes{v2.0k}{1994/05/21}{Use \cmd{\aalph}}
%    \begin{macrocode}
\def\docincludeaux{%
  \def\thepart{\aalph{part}}\def\filesep{\thepart-}%
  \let\filekey\@gobble
  \g@addto@macro\index@prologue{%
    \gdef\@oddfoot{\parbox{\textwidth}{\strut\footnotesize
       \raggedright{\bfseries File Key:} \filekey}}%
    \let\@evenfoot\@oddfoot}%
  \global\let\docincludeaux\relax
 \gdef\@oddfoot{%
   \expandafter\ifx\csname ver@\currentfile\endcsname\relax
    File \thepart: {\ttfamily\currentfile} %
   \else
    \GetFileInfo{\currentfile}%
    File \thepart: {\ttfamily\filename} %
    Date: \filedate\ %
    Version \fileversion
    \fi
    \hfill\thepage}%
 \let\@evenfoot\@oddfoot}%
%    \end{macrocode}
% \end{macro}
%
%    \begin{macrocode}
%<*platex>
\providecommand*{\file}[1]{\texttt{#1}}
\providecommand*{\pstyle}[1]{\textsl{#1}}
\providecommand*{\Lcount}[1]{\textsl{\small#1}}
\providecommand*{\Lopt}[1]{\textsf{#1}}
\providecommand\dst{{\normalfont\scshape docstrip}}
\providecommand\NFSS{\textsf{NFSS}}
%
\newcounter{@clineno}
\def\mlineplus#1{\setcounter{@clineno}{\arabic{CodelineNo}}%
   \addtocounter{@clineno}{#1}\arabic{@clineno}}

\def\jtechI{�����ܸ�\TeX �ƥ��˥���֥å��ɡ�}
\def\tlatexbook{�ؽ����б��� �ѡ����ʥ����ܸ�\TeX ���ȥ�ե���󥹥����ɡ�}
%
\def\sample#1{%
  \hbox to\linewidth\bgroup\vrule width.1pt\hss
    \vbox\bgroup\hrule height.1pt
      \vskip.5\baselineskip
      \vbox to\linewidth\bgroup\tate\hsize=#1\relax\vss}
\def\endsample{%
      \vss\egroup
      \vskip.5\baselineskip
    \hrule height.1pt\egroup
  \hss\vrule width.1pt\egroup}
%</platex>
%    \end{macrocode}
%
%    \begin{macrocode}
\def\task#1#2{}
%</class>
%    \end{macrocode}
% \Finale
%
